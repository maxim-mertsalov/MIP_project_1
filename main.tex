% Metódy inžinierskej práce

\documentclass[10pt,twoside,slovak,a4paper]{coursepaper}

\usepackage[slovak]{babel}
%\usepackage[T1]{fontenc}
\usepackage[IL2]{fontenc} % lepšia sadzba písmena Ľ než v T1
\usepackage[utf8]{inputenc}
\usepackage{graphicx}
\usepackage{url} % príkaz \url na formátovanie URL
\usepackage{hyperref} % odkazy v texte budú aktívne (pri niektorých triedach dokumentov spôsobuje posun textu)

\usepackage{cite}
%\usepackage{times}

\pagestyle{headings}

\title{Deep Learning vs. Machine Learning in digital marketing \thanks{Semestrálny projekt v predmete Metódy inžinierskej práce, ak. rok 2024/25, vedenie: Yevheniia Kataieva}} % meno a priezvisko vyučujúceho na cvičeniach

\author{Maksym Mertsalov\\[2pt]
	{\small Slovenská technická univerzita v Bratislave}\\
	{\small Fakulta informatiky a informačných technológií}\\
	{\small \texttt{xmertsalov@stuba.sk}}
	}

\date{\small 09. september 2024} % upravte



\begin{document}

\maketitle

\begin{abstract}
With the huge development of online advertising, it is becoming more and more difficult to monitor it. There are difficulties in matching human interests, blocking unnecessary traffic, displaying popular queries, etc. Therefore, in this paper, based on artificial intelligence technologies such as natural language processing, pattern recognition, machine learning, deep learning and data tracking, we will develop an online advertising monitoring system to improve the recognition performance, create a unified database and data prediction system. This article will contribute to the optimization and upgrading of advertising systems and the development of the Internet advertising industry.\cite{10314235}
\ldots
\end{abstract}



\section{Introduction(first small version)}
\begingroup
Since the first brands emerged more than two centuries ago, advertising has undergone significant changes. Companies used to use newspapers, magazines, flyers, billboards and marketing calls to spread their messages. Later, they moved to mass media: television and radio.\par

However, digital technology is facilitating the emergence of a third sector: the online space. Search engines, websites and social media make advertising more ubiquitous than ever. \par

Artificial intelligence has become a real force, changing the landscape once again. Bringing higher intelligence to the digital world is enabling advances that were impossible when humans controlled the show. But artificial intelligence is only a small part of what's behind the logic of advertising in big services. Let's discuss machine learning, deep learning, artificial intelligence and more in the digital world below.\par

\endgroup

\section{Some test paragraph}
This is TEXT and this too \newline
Input: \input{new.tex}


\section{Nejaká časť} \label{nejaka}

Z obr.~\ref{f:rozhod} je všetko jasné. 

\begin{figure*}[tbh]
\centering
%\includegraphics[scale=1.0]{diagram.pdf}
Aj text môže byť prezentovaný ako obrázok. Stane sa z neho označný plávajúci objekt. Po vytvorení diagramu zrušte znak \texttt{\%} pred príkazom \verb|\includegraphics| označte tento riadok ako komentár (tiež pomocou znaku \texttt{\%}).
\caption{Rozhodujúci argument.}
\label{f:rozhod}
\end{figure*}



\section{Iná časť} \label{ina}

Základným problémom je teda\ldots{} Najprv sa pozrieme na nejaké vysvetlenie (časť~\ref{ina:nejake}), a potom na ešte nejaké (časť~\ref{ina:nejake}).\footnote{Niekedy môžete potrebovať aj poznámku pod čiarou.}

Môže sa zdať, že problém vlastne nejestvuje\cite{Coplien:MPD}, ale bolo dokázané, že to tak nie je~\cite{Czarnecki:Staged, Czarnecki:Progress}. Napriek tomu, aj dnes na webe narazíme na všelijaké pochybné názory\cite{PLP-Framework}. Dôležité veci možno \emph{zdôrazniť kurzívou}.


\subsection{Nejaké vysvetlenie} \label{ina:nejake}

Niekedy treba uviesť zoznam:

\begin{itemize}
\item jedna vec
\item druhá vec
	\begin{itemize}
	\item x
	\item y
	\end{itemize}
\end{itemize}

Ten istý zoznam, len číslovaný:

\begin{enumerate}
\item jedna vec
\item druhá vec
	\begin{enumerate}
	\item x
	\item y
	\end{enumerate}
\end{enumerate}


\subsection{Ešte nejaké vysvetlenie} \label{ina:este}

\paragraph{Veľmi dôležitá poznámka.}
Niekedy je potrebné nadpisom označiť odsek. Text pokračuje hneď za nadpisom.



\section{Dôležitá časť} \label{dolezita}




\section{Ešte dôležitejšia časť} \label{dolezitejsia}




\section{Záver} \label{zaver} % prípadne iný variant názvu



%\acknowledgement{Ak niekomu chcete poďakovať\ldots}


% týmto sa generuje zoznam literatúry z obsahu súboru literatura.bib podľa toho, na čo sa v článku odkazujete
\bibliography{literatura}
\bibliographystyle{abbrv} % prípadne alpha, abbrv alebo hociktorý iný. default: plain
\end{document}
